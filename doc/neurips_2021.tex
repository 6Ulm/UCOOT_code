\documentclass{article}

% if you need to pass options to natbib, use, e.g.:
%     \PassOptionsToPackage{numbers, compress}{natbib}
% before loading neurips_2021

% ready for submission
\usepackage{neurips_2021}

% to compile a preprint version, e.g., for submission to arXiv, add add the
% [preprint] option:
%     \usepackage[preprint]{neurips_2021}

% to compile a camera-ready version, add the [final] option, e.g.:
%     \usepackage[final]{neurips_2021}

% to avoid loading the natbib package, add option nonatbib:
%    \usepackage[nonatbib]{neurips_2021}

\usepackage[utf8]{inputenc} % allow utf-8 input
\usepackage[T1]{fontenc}    % use 8-bit T1 fonts
\usepackage{hyperref}       % hyperlinks
\usepackage{url}            % simple URL typesetting
\usepackage{booktabs}       % professional-quality tables
\usepackage{amsfonts}       % blackboard math symbols
\usepackage{nicefrac}       % compact symbols for 1/2, etc.
\usepackage{microtype}      % microtypography
\usepackage{xcolor}         % colors
\usepackage{graphicx}

\usepackage{caption}
\usepackage{algorithm}
\usepackage{algorithmic}
\usepackage{enumitem}
\usepackage{amsmath,amssymb,amsthm}
\usepackage{thmtools}
\usepackage{cleveref}

\newtheorem{theorem}{Theorem}[section]
\newtheorem{lemma}[theorem]{Lemma}
\newtheorem{proposition}[theorem]{Proposition}
\newtheorem{corollary}[theorem]{Corollary}
\newtheorem{definition}{Definition}[section]

\theoremstyle{remark}
\newtheorem{remark}[theorem]{Remark}
\renewcommand{\qedsymbol}{$\blacksquare$}

\title{Gromov-Monge distance helps understand the Gromov-Wasserstein distance}

% The \author macro works with any number of authors. There are two commands
% used to separate the names and addresses of multiple authors: \And and \AND.
%
% Using \And between authors leaves it to LaTeX to determine where to break the
% lines. Using \AND forces a line break at that point. So, if LaTeX puts 3 of 4
% authors names on the first line, and the last on the second line, try using
% \AND instead of \And before the third author name.

\author{
  % Quang Huy Tran \thanks{Use footnote for providing further information
  %   about author (webpage, alternative address)---\emph{not} for acknowledging
  %   funding agencies.} \\
  Quang Huy Tran \\
  Univ. Bretagne-Sud, CNRS, IRISA \\
  F-56000 Vannes \\
  \texttt{quang-huy.tran@univ-ubs.fr}
}

\begin{document}

\maketitle

\paragraph{Introduction.} Let $C$ be a $4$-D tensor, where $C_{ijkl} = \vert C^x_{ik} - C^y_{jl} \vert^p$, for $p \geq 1$. The product measure $\mu^{\otimes 2} := \mu \otimes \mu$, where $(\mu \otimes \nu)_{ij} = \mu_i \otimes \nu_j$.

For $\mathcal X = (C_x, \mu_x)$ and $\mathcal Y = (C_y, \mu_y)$, where $C_x \in \mathbb R^{m \times m}, C_y \in \mathbb R^{n \times n}$, and $\mu_x \in \mathbb R^{m}_{\geq 0}, \mu_y \in \mathbb R^{n}_{\geq 0}$,
define the UCOOT's objective function: for $\rho_1, \rho_2 \geq 0$ and $P,Q \geq 0$,
\begin{equation}
    G_{C, \rho_{12}}(P,Q) = \langle C, P \otimes Q \rangle + \rho_1 \text{KL}(P_{\# 1} \otimes Q_{\# 1} \vert \mu_x \otimes \mu_x) + \rho_2 \text{KL}(P_{\# 2} \otimes Q_{\# 2} \vert \mu_y \otimes \mu_y)
\end{equation}
The UGW reads
\begin{equation*}
    \begin{split}
     \text{UGW}_{\rho_{12}}(\mathcal X, \mathcal Y) &= \inf_{P \geq 0} G_{C,\rho_{12}}(P, P) = \inf_{\substack{P, Q \geq 0 \\ P = Q}} G_{C,\rho_{12}}(P, Q) \\
        &\geq \inf_{P,Q \geq 0} G_{C,\rho_{12}}(P, Q) = \inf_{\substack{P, Q \geq 0 \\ m(P) = m(Q)}} G_{C,\rho_{12}}(P, Q) =  \text{LB-UGW}_{\rho_{12}}(\mathcal X, \mathcal Y)
    \end{split}
\end{equation*}
Let $D$ be a $m \times n$ matrix whose coordinates are distances between features. Define the unbalanced OT's objective function: for $P \geq 0$,
\begin{equation}
    F_{D, \rho_{34}}(P) = \langle D, P \rangle + \rho_3 \text{KL}(P_{\# 1} \vert \mu_x) + \rho_4 \text{KL}(P_{\# 2} \vert \mu_y)
\end{equation}
and the UOT reads
\begin{equation*}
    \text{UOT}_{\rho_{34}}(\mu_x,\mu_y) = \inf_{P \geq 0} F_{D, \rho_{34}}(P)
\end{equation*}
The FGW reads: for $\lambda \in [0,1]$,
\begin{equation*}
    \text{FGW}_{\lambda}(\mathcal X, \mathcal Y) = \inf_{P \in U(\mu_x, \mu_y)} \lambda \langle C, P \otimes P \rangle + (1 - \lambda) \langle D, P \rangle
\end{equation*}
\paragraph{Formulation.} Now, fused UGW reads
\begin{equation}
    \begin{split}
      \text{FUGW}_{\rho, \lambda}(\mathcal X, \mathcal Y) &= \inf_{P \geq 0} \lambda G_{C,\rho_{12}}(P, P) + (1 - \lambda) F_{D,\rho_{34}}(P) \\
        &= \inf_{\substack{P,Q \geq 0 \\ P = Q}} \lambda G_{C,\rho_{12}}(P,Q) + \frac{1 - \lambda}{2} \big[ F_{D,\rho_{34}}(P) + F_{D,\rho_{34}}(Q) \big]
    \end{split}
\end{equation}
\begin{remark}
    When $\rho_1, \rho_2, \rho_3, \rho_4 \to \infty$, then we recover FGW. When $\rho_1 = \rho_3 = \rho_4 = 0$, and either $\rho_2 = \infty$, then we recover semi-relaxed FGW.
\end{remark}
Estimating UGW is numerically difficult, let alone FUGW. Thus, we study its lower bound:
\begin{equation}
    \label{lbfugw}
    \begin{split}
       \text{LB-FUGW}_{\rho, \lambda}(\mathcal X, \mathcal Y) &= \inf_{\substack{P,Q \geq 0 \\ m(P) = m(Q)}} \lambda G_{C,\rho_{12}}(P,Q) + \frac{1 - \lambda}{2} \big[ F_{D,\rho_{34}}(P) + F_{D,\rho_{34}}(Q) \big]
    \end{split}
\end{equation}
The additional mass constraint $m(P) = m(Q)$ may be advantageous because it may help BCD algo more numerically stable, similar to the UGW.
\begin{proposition}
  The problem \ref{lbfugw} admits a minimiser under which condition? Should be similar to UGW and UOT.
\end{proposition}
\begin{proposition}
    \label{interpol_fugw}
    (Interpolation property)
    \begin{itemize}
        \item Intuitively, $\text{FUGW}_{\rho, \lambda}(\mathcal X, \mathcal Y)$ converges to $\text{UGW}_{\rho_{12}}(\mathcal X, \mathcal Y)$, when $\lambda \to 1$, and to $\text{UOT}_{\rho_{34}}(\mu_x, \mu_y)$ when $\lambda \to 0$.
        
        \item $\text{LB-FUGW}_{\rho, \lambda}(\mathcal X, \mathcal Y)$ converges to $\text{LB-UGW}_{\rho_{12}}(\mathcal X, \mathcal Y)$ when $\lambda \to 1$, and to $\text{UOT}_{\rho_{34}}(\mu_x, \mu_y)$ when $\lambda \to 0$.
    \end{itemize}
\end{proposition}
\begin{proof}
When $\lambda \to 1$, then same proof as the FUGW. When $\lambda \to 0$, consider the following lower bound
\begin{equation*}
    \begin{split}
      \widetilde{\text{LB-FUGW}}_{\rho, \lambda}(\mathcal X, \mathcal Y) &= \inf_{P,Q \geq 0} \lambda G_{C,\rho_{12}}(P,Q) + \frac{1 - \lambda}{2} \big[ F_{D,\rho_{34}}(P) + F_{D,\rho_{34}}(Q) \big]
    \end{split}
\end{equation*}
Clearly, $\text{FUGW} \geq \text{LB-FUGW} \geq \widetilde{\text{LB-FUGW}}$. When $\lambda \to 0$, show that $\widetilde{\text{LB-FUGW}} \to \frac{1}{2}(\text{UOT}+ \text{UOT}) = \text{UOT}$ (intuitively, this should be true). By sandwich theorem and proposition \ref{interpol_fugw}, we conclude that $\text{LB-FUGW} \to \text{UOT}$ when $\lambda \to 0$.
This is interesting because despite the mass constraint in the $\text{LB-FUGW}$, it has virtually no impact on the two UOT terms, for small $\lambda$.
\end{proof}
\begin{proposition}
    For fixed $\lambda \in [0,1]$, for every $\rho_1, \rho_2, \rho_3, \rho_4 > 0$, we have
    \begin{itemize}
        \item $\text{FUGW}_{\rho, \lambda}(\mathcal X, \mathcal Y) \leq \text{FGW}_{\lambda}(\mathcal X, \mathcal Y)$. Furthermore, $\text{FUGW}_{\rho, \lambda}(\mathcal X, \mathcal Y) = 0$ iff $\text{FGW}_{\lambda}(\mathcal X, \mathcal Y) = 0$.

        \item $\text{LB-FUGW}_{\rho, \lambda}(\mathcal X, \mathcal Y) \leq \text{LB-FGW}_{\lambda}(\mathcal X, \mathcal Y)$. Furthermore, $\text{LB-FUGW}_{\rho, \lambda}(\mathcal X, \mathcal Y) = 0$ iff $\text{LB-FGW}_{\lambda}(\mathcal X, \mathcal Y) = 0$.
    \end{itemize}
    
\end{proposition}

\paragraph{Entropic LB-UGW setting.} Two possible entropic regularisation versions
\begin{enumerate}
    \item Following UGW (corresponding to \texttt{reg\_mode = "joint"}).
    \begin{equation}
        \label{ent_ugw}
       \text{LB-FUGW}_{\varepsilon, \rho, \lambda}(\mathcal X, \mathcal Y) = \inf_{\substack{P,Q \geq 0 \\ m(P) = m(Q)}} H_{\rho, \lambda}(P,Q) + \varepsilon \text{KL}\big( P \otimes Q \vert (\mu_x \otimes \mu_y)^{\otimes 2} \big)
    \end{equation}
    
    \item Following COOT (corresponding to \texttt{reg\_mode = "independent"})
    \begin{equation}
        \label{ent_coot}
       \text{LB-FUGW}_{\varepsilon, \rho, \lambda}(\mathcal X, \mathcal Y) = \inf_{\substack{P,Q \geq 0 \\ m(P) = m(Q)}} H_{\rho, \lambda}(P,Q) + \varepsilon \text{KL}( P \vert \mu_x \otimes \mu_y) + \varepsilon \text{KL}( Q \vert \mu_x \otimes \mu_y)
    \end{equation}    
\end{enumerate}
\begin{proposition}
    In both versions, we have 
    $\text{LB-FUGW}_{\varepsilon, \rho, \lambda}(\mathcal X, \mathcal Y) \to \text{LB-FUGW}_{\rho, \lambda}(\mathcal X, \mathcal Y)$, 
    when $\varepsilon \to 0$.
\end{proposition}

The previous formulation is nice in terms of theoritical properties but bad in terms of implementation because it introduces too many hyperparameters (coming from the UOT term). It may be enough to relax the mass via the UGW term, no need to further introduce in the UOT term. Only linear terms are kept.
\begin{equation*}
    \begin{split}
     \text{FUGW}_{\rho}(\mathcal X, \mathcal Y) = \inf_{\substack{P, Q \geq 0 \\ m(P) = m(Q)}} G_{\rho}(P, Q) + \lambda \langle D, P+Q \rangle
    \end{split}
\end{equation*}
Few observations:
\begin{enumerate}
    \item If $\rho_1 = \rho_2 = \infty$, then recover fused GW.
    \item FUGW is robust to outliers (the proof should be similar to that of UGW).
    \item With the usual choice of cost $C$, we have
    \begin{equation*}
    \begin{split}
        \text{FUGW}_{\rho}(\mathcal X, \mathcal Y) &= \inf_{\substack{P, Q \geq 0 }} G_{\rho}(P, Q) + \lambda \langle D, P+Q \rangle \\
        &= \inf_{\substack{P \geq 0 }} G_{\rho}(P, P) + 2 \lambda \langle D, P \rangle
    \end{split}
    \end{equation*}
\end{enumerate}

%%%%%%%%%%%%%%%%%%%%%%%
\newpage
In practice, we consider
\begin{equation}
    \label{lbfugw2}
    \begin{split}
       \text{FUCOOT}_{\rho, \lambda}(\mathcal X, \mathcal Y) &= \inf_{\substack{P,Q \geq 0 \\ m(P) = m(Q)}} G_{C,\rho_{xy}}(P,Q) + \lambda_s F_{D_s,\rho_{s}}(P) + \lambda_f F_{D_f,\rho_{f}}(Q)
    \end{split}
\end{equation}
Under the constraint $m(P) = m(Q) = m$, the complete objective function of FUCOOT reads
\begin{equation*}
    \begin{split}
        H_{\rho, \lambda}(P,Q) &= G_{C,\rho_{xy}}(P,Q) + \lambda_s F_{D_s,\rho_{s}}(P) + \lambda_f F_{D_f,\rho_{f}}(Q) \\
        &= \langle C, P \otimes Q \rangle
        + \rho_x \text{KL}(P_{\# 1} \otimes Q_{\# 1} \vert \mu_{nx} \otimes \mu_{dx}) 
        + \rho_y \text{KL}(P_{\# 2} \otimes Q_{\# 2} \vert \mu_{ny} \otimes \mu_{dy}) \\
        &+ \lambda_s \Big( \langle D_s, P \rangle + \rho_1^{(s)} \text{KL}(P_{\# 1} \vert \mu_{nx}) + \rho_2^{(s)} \text{KL}(P_{\# 2} \vert \mu_{ny}) \Big) \\
        &+ \lambda_f \Big( \langle D_f, Q \rangle + \rho_1^{(f)} \text{KL}(Q_{\# 1} \vert \mu_{dx}) + \rho_2^{(f)} \text{KL}(Q_{\# 2} \vert \mu_{dy}) \Big)
    \end{split}
\end{equation*}
Two possible entropic regularisation versions, define $\mu_n := \mu_{nx} \otimes \mu_{ny}$ and $\mu_d := \mu_{dx} \otimes \mu_{dy}$.
\begin{enumerate}
    \item Following UGW (corresponding to \texttt{reg\_mode = "joint"}).
    \begin{equation}
        \label{ent_ugw}
       \text{LB-FUGW}_{\varepsilon, \rho, \lambda}(\mathcal X, \mathcal Y) = \inf_{\substack{P,Q \geq 0 \\ m(P) = m(Q)}} H_{\rho, \lambda}(P,Q) + \varepsilon \text{KL}( P \otimes Q \vert \mu_n \otimes \mu_d)
    \end{equation}
    
    \item Following COOT (corresponding to \texttt{reg\_mode = "independent"})
    \begin{equation}
        \label{ent_coot}
       \text{LB-FUGW}_{\varepsilon, \rho, \lambda}(\mathcal X, \mathcal Y) = \inf_{\substack{P,Q \geq 0 \\ m(P) = m(Q)}} H_{\rho, \lambda}(P,Q) + \varepsilon_s \text{KL}( P \vert \mu_n) + \varepsilon_f \text{KL}( Q \vert \mu_d)
    \end{equation}    
\end{enumerate}
Using the relation 
\begin{equation*}
    \text{KL}(P \vert \mu) = \int \log \big( \frac{dP}{d\mu}\big) dP - m(P) + m(\mu)
\end{equation*}
and for fixed $P$, denote $m = m(P)$.
\begin{equation*}
    \begin{split}
        &\text{KL}(P_{\# 1} \otimes Q_{\# 1} \vert \mu \otimes \nu) \\
        &= m(Q) \text{KL}(P_{\# 1} \vert \mu) + m(P) \text{KL}(Q_{\# 1} \vert \nu) + \big[m(P) - m(\mu)\big] \big[m(Q) - m(\nu)\big] \\
        &= \int \Big( \int \log \big(\frac{d P_{\# 1}}{d \mu} \big) dP_{\# 1} \Big) dQ + m(P) \text{KL}(Q_{\# 1} \vert \nu) -
        m(\nu)\big[m(P) - m(\mu)\big] \\
        &= \int \Big( \int \log \big(\frac{d P_{\# 1}}{d \mu} \big) dP_{\# 1} \Big) dQ + m \text{KL}(Q_{\# 1} \vert \nu) + \text{ constant}
    \end{split}
\end{equation*}
Then, solving the problem \ref{ent_ugw} is equivalent to solving
\begin{equation*}
    \inf_{Q \geq 0} \langle L, Q \rangle + \Big( m \rho_x + \lambda_f \rho_1^{(f)} \Big) \text{KL}(Q_{\# 1} \vert \mu_{dx}) + \Big( m \rho_y + \lambda_f \rho_2^{(f)} \Big) \text{KL}(Q_{\# 2} \vert \mu_{dy}) + \varepsilon {\color{red}{m}} \text{KL}(Q \vert \mu_d)
\end{equation*}
where
\begin{equation*}
    L = C \otimes P + \lambda_f D_f + \rho_x \langle \log \big(\frac{P_{\# 1}}{\mu_{nx}} \big), P_{\# 1} \rangle + \rho_y \langle \log \big(\frac{P_{\# 2}}{\mu_{ny}} \big), P_{\# 2} \rangle + \color{red}{\varepsilon \langle \log \big(\frac{P}{\mu_n} \big), P \rangle}
\end{equation*}

and olving the problem \ref{ent_coot} is equivalent to solving
\begin{equation*}
    \inf_{Q \geq 0} \langle L, Q \rangle + \Big( m \rho_x + \lambda_f \rho_1^{(f)} \Big) \text{KL}(Q_{\# 1} \vert \mu_{dx}) + \Big( m \rho_y + \lambda_f \rho_2^{(f)} \Big) \text{KL}(Q_{\# 2} \vert \mu_{dy})  + \varepsilon_f \text{KL}(Q \vert \mu_d)
\end{equation*}
where
\begin{equation*}
    L = C \otimes P + \lambda_f D_f + \rho_x \langle \log \big(\frac{P_{\# 1}}{\mu_{nx}} \big), P_{\# 1} \rangle + \rho_y \langle \log \big(\frac{P_{\# 2}}{\mu_{ny}} \big), P_{\# 2} \rangle
\end{equation*}

\newpage
\begin{algorithm}[H]
  \caption{Generic scaling algorithm}
  \textbf{Input.} Solving
  \begin{equation*}
      \min_{P \geq 0} \; \langle C, P \rangle + \rho_1 \text{KL}(P_{\# 1} \vert \mu) + \rho_2 \text{KL}(P_{\# 2} \vert \nu) + \varepsilon \text{KL}(P \vert \mu \otimes \nu)
  \end{equation*}.

  \textbf{Output.} Pair of optimal dual vectors $(f,g)$ and coupling $P$.
  \begin{enumerate}
      \item While not converge, update
      \begin{equation*}
          \begin{cases}
            f = -\frac{\rho_1}{\rho_1 + \varepsilon} \log \sum_j \exp \big( g_j + \log \nu_j - \frac{C_{\cdot,j}}{\varepsilon} \big) \\
            g = -\frac{\rho_2}{\rho_2 + \varepsilon} \log \sum_i \exp \big( f_i + \log \mu_i - \frac{C_{i,\cdot}}{\varepsilon} \big)
          \end{cases}
      \end{equation*}
      \item Calculate $P = (\mu \otimes \nu) \exp \big(f \oplus g - \frac{C}{\varepsilon} \big)$.
  \end{enumerate}
  \label{algo:scaling}
\end{algorithm}
Here $\otimes$ and $\oplus$ are the tensor product and sum, respectively. Some tricks: for any matrix $M$, we write $M^{\odot 2} := M \odot M$, where $\odot$ is the element-wise multiplication.
\begin{enumerate}
    \item Suppose $A \in \mathbb R^{n_1 \times d_1}$ and $B \in \mathbb R^{n_2 \times d_2}$. For $P \in \mathbb R^{d_1 \times d_2}$, we have  $\vert A - B \vert^2 \otimes P \in \mathbb R^{n_1 \times n_2}$, where
    \begin{equation*}
      \vert A - B \vert^2 \otimes P = A^{\odot 2} P_{\# 1} \oplus B^{\odot 2} P_{\# 2} - 2 A P B^T.
    \end{equation*}
    
    \item If $A = (a_1,...,a_m) \in \mathbb R^{m \times d}$ and $B = (b_1,...,b_n) \in \mathbb R^{n \times d}$, then the matrix $D \in \mathbb R^{m \times n}$ defined by $D_{ij} = \vert\vert a_i - b_j \vert\vert_2^2$ can be decomposed as $D = D_a D_b^T$, where $D_a = (A^{\odot 2} 1_d, 1_m, -\sqrt{2} A) \in \mathbb R^{m \times (d+2)}$ and
    $D_b = (1_n, B^{\odot 2} 1_d, \sqrt{2} B) \in \mathbb R^{n \times (d+2)}$. So, instead of storing $D$, we store $D_a$ and $D_b$, so that we can scale up easily when the dimension $d$ is small.
\end{enumerate}
So, for $C = \vert C_x - C_y \vert^2$, with $(C_x)_{ij} = \vert\vert x_i - x_j \vert\vert_2^2$ and 
$(C_y)_{kl} = \vert\vert y_k - y_l \vert\vert_2^2$ and 
$D_{ij} = \vert\vert a^{(x)}_i - a^{(y)}_j \vert\vert_2^2$, we have
$C_x P C_y^T = A_1 A_2^T P B_2 B_1^T$. Denote $M = A_2^T P B_2 \in \mathbb R^{(d_1+2) \times (d_2+2)}$, then $C_x P C_y^T = A_1 M B_1^T$.
\begin{algorithm}[H]
    \caption{Approximation algorithm for FUGW}
    \textbf{Input.} Graphs $X = (C^x, \mu_x), Y = (C^y, \mu_y)$, with distance matrix $D$ between features, parameters $\rho_1, \rho_2 > 0$, interpolation parameter $\lambda \in [0,1]$, 
    the regularisation parameter $\varepsilon > 0$ and initialisation $P_0$.
  
    \textbf{Output.} Pair of optimal couplings $(P,Q)$.
    \begin{itemize}
      \item \textbf{While} $P_k$ has not converged \textbf{do}
      \begin{enumerate}
        \item $Q_{k+1}$ is the solution for fixed $P_k$.
        \item Rescale $Q_{k+1} = \sqrt{\frac{m(P_k)}{m(Q_{k+1})}} Q_{k+1}$.

        \item $P_{k+1}$ is the solution for fixed $Q_{k+1}$.

        \item Rescale $P_{k+1} = \sqrt{\frac{m(Q_{k+1})}{m(P_{k+1})}} P_{k+1}$.
      \end{enumerate}
    \end{itemize}
    \label{algo:algo1}
\end{algorithm}
The regularised and unregularised UOT can be solved with MM algorithm: the iteration reads
\begin{equation}
    \begin{split}
        P &= \left[ \left( \frac{\mu}{P_{\# 1}}\right)^{\lambda_1} \otimes \left( \frac{\nu}{P_{\# 2}}\right)^{\lambda_2} \right] \odot 
    P^{\lambda_1 + \lambda_2} \odot (\mu \otimes v)^r \odot \exp\left(-\frac{C}{\lambda} \right) \\
    &= \frac{P^{\lambda_1 + \lambda_2}}{P_{\# 1}^{\lambda_1} \otimes P_{\# 2}^{\lambda_2}} 
    \odot \left( \mu^{\lambda_1 + r} \otimes \nu^{\lambda_2 + r} \right) 
    \odot \exp\left(-\frac{C}{\lambda} \right)
    \end{split}
\end{equation}
where $\lambda = \rho_1 + \rho_2 + \varepsilon$ and 
$\lambda_i = \frac{\rho_i}{\lambda}$ and $r = \frac{\varepsilon}{\lambda}$. Or for more stability, 
\begin{equation}
    \begin{split}
        \log P &= (\lambda_1 + \lambda_2) \log P - (\lambda_1 \log P_{\# 1} \oplus \lambda_2 \log P_{\# 2}) \\
        &+ \left[ (\lambda_1 + r) \log \mu \oplus (\lambda_2 + r) \log \nu \right] -\frac{C}{\lambda}
    \end{split}
\end{equation}

In the example of neuro-image: source and target data
\begin{itemize}
    \item Functional data: $F_s \in \mathbb R^{160k \times 300}, F_t \in \mathbb R^{60k \times 300}$.
    \item Anatomy data: $A_s \in \mathbb R^{160k \times 6}, A_t \in \mathbb R^{60k \times 6}$.
\end{itemize}
Input of FUGW: distance matrix $K \in \mathbb R^{160k \times 60k}$ between $A_s$ and $A_t$ for the fused part. For GW part: distance matrix $D_s \in \mathbb R^{160k \times 160k}$ and $D_t \in \mathbb R^{60k \times 60k}$.

Formulation used in fugw full
\begin{equation}
    \begin{split}
        \text{FUGW}_{\rho, \alpha}(X, Y) 
        &= \min_{P, Q \geq 0} \langle \text{cost}, P \otimes Q \rangle \\
        &+ \rho_1 \text{KL}(P_{\# 1} \otimes Q_{\# 1} | \mu \otimes \mu) +
        \rho_2 \text{KL}(P_{\# 2} \otimes Q_{\# 2} | \nu \otimes \nu) \\
        &+ \alpha \left[ \langle K, P \rangle + \rho_3 \text{KL}(P_{\# 1} | \mu) + 
        \rho_4 \text{KL}(P_{\# 2} | \nu) \right] \\
        &+ \alpha \left[ \langle K, Q \rangle + \rho_3 \text{KL}(Q_{\# 1} | \mu) + 
        \rho_4 \text{KL}(Q_{\# 2} | \nu) \right]
    \end{split}
\end{equation}

%%%%%%%%%%%%%%%%%%%%%%%%%%%%%%%%%%%%%%%
\bibliographystyle{plainnat}
\bibliography{neurips_2021.bib}
% \section{Submission of papers to NeurIPS 2021}

% Please read the instructions below carefully and follow them faithfully.

% \subsection{Style}

% Papers to be submitted to NeurIPS 2021 must be prepared according to the
% instructions presented here. Papers may only be up to {\bf nine} pages long,
% including figures. Additional pages \emph{containing only acknowledgments and
% references} are allowed. Papers that exceed the page limit will not be
% reviewed, or in any other way considered for presentation at the conference.

% The margins in 2021 are the same as those in 2007, which allow for $\sim$$15\%$
% more words in the paper compared to earlier years.

% Authors are required to use the NeurIPS \LaTeX{} style files obtainable at the
% NeurIPS website as indicated below. Please make sure you use the current files
% and not previous versions. Tweaking the style files may be grounds for
% rejection.

% \subsection{Retrieval of style files}

% The style files for NeurIPS and other conference information are available on
% the World Wide Web at
% \begin{center}
%   \url{http://www.neurips.cc/}
% \end{center}
% The file \verb+neurips_2021.pdf+ contains these instructions and illustrates the
% various formatting requirements your NeurIPS paper must satisfy.

% The only supported style file for NeurIPS 2021 is \verb+neurips_2021.sty+,
% rewritten for \LaTeXe{}.  \textbf{Previous style files for \LaTeX{} 2.09,
%   Microsoft Word, and RTF are no longer supported!}

% The \LaTeX{} style file contains three optional arguments: \verb+final+, which
% creates a camera-ready copy, \verb+preprint+, which creates a preprint for
% submission to, e.g., arXiv, and \verb+nonatbib+, which will not load the
% \verb+natbib+ package for you in case of package clash.

% \paragraph{Preprint option}
% If you wish to post a preprint of your work online, e.g., on arXiv, using the
% NeurIPS style, please use the \verb+preprint+ option. This will create a
% nonanonymized version of your work with the text ``Preprint. Work in progress.''
% in the footer. This version may be distributed as you see fit. Please \textbf{do
%   not} use the \verb+final+ option, which should \textbf{only} be used for
% papers accepted to NeurIPS.

% At submission time, please omit the \verb+final+ and \verb+preprint+
% options. This will anonymize your submission and add line numbers to aid
% review. Please do \emph{not} refer to these line numbers in your paper as they
% will be removed during generation of camera-ready copies.

% The file \verb+neurips_2021.tex+ may be used as a ``shell'' for writing your
% paper. All you have to do is replace the author, title, abstract, and text of
% the paper with your own.

% The formatting instructions contained in these style files are summarized in
% Sections \ref{gen_inst}, \ref{headings}, and \ref{others} below.

% \section{General formatting instructions}
% \label{gen_inst}

% The text must be confined within a rectangle 5.5~inches (33~picas) wide and
% 9~inches (54~picas) long. The left margin is 1.5~inch (9~picas).  Use 10~point
% type with a vertical spacing (leading) of 11~points.  Times New Roman is the
% preferred typeface throughout, and will be selected for you by default.
% Paragraphs are separated by \nicefrac{1}{2}~line space (5.5 points), with no
% indentation.

% The paper title should be 17~point, initial caps/lower case, bold, centered
% between two horizontal rules. The top rule should be 4~points thick and the
% bottom rule should be 1~point thick. Allow \nicefrac{1}{4}~inch space above and
% below the title to rules. All pages should start at 1~inch (6~picas) from the
% top of the page.

% For the final version, authors' names are set in boldface, and each name is
% centered above the corresponding address. The lead author's name is to be listed
% first (left-most), and the co-authors' names (if different address) are set to
% follow. If there is only one co-author, list both author and co-author side by
% side.

% Please pay special attention to the instructions in Section \ref{others}
% regarding figures, tables, acknowledgments, and references.

% \section{Headings: first level}
% \label{headings}

% All headings should be lower case (except for first word and proper nouns),
% flush left, and bold.

% First-level headings should be in 12-point type.

% \subsection{Headings: second level}

% Second-level headings should be in 10-point type.

% \subsubsection{Headings: third level}

% Third-level headings should be in 10-point type.

% \paragraph{Paragraphs}

% There is also a \verb+\paragraph+ command available, which sets the heading in
% bold, flush left, and inline with the text, with the heading followed by 1\,em
% of space.

% \section{Citations, figures, tables, references}
% \label{others}

% These instructions apply to everyone.

% \subsection{Citations within the text}

% The \verb+natbib+ package will be loaded for you by default.  Citations may be
% author/year or numeric, as long as you maintain internal consistency.  As to the
% format of the references themselves, any style is acceptable as long as it is
% used consistently.

% The documentation for \verb+natbib+ may be found at
% \begin{center}
%   \url{http://mirrors.ctan.org/macros/latex/contrib/natbib/natnotes.pdf}
% \end{center}
% Of note is the command \verb+\citeppt+, which produces citations appropriate for
% use in inline text.  For example,
% \begin{verbatim}
%    \citeppt{hasselmo} investigated\dots
% \end{verbatim}
% produces
% \begin{quote}
%   Hasselmo, et al.\ (1995) investigated\dots
% \end{quote}

% If you wish to load the \verb+natbib+ package with options, you may add the
% following before loading the \verb+neurips_2021+ package:
% \begin{verbatim}
%    \PassOptionsToPackage{options}{natbib}
% \end{verbatim}

% If \verb+natbib+ clashes with another package you load, you can add the optional
% argument \verb+nonatbib+ when loading the style file:
% \begin{verbatim}
%    \usepackage[nonatbib]{neurips_2021}
% \end{verbatim}

% As submission is double blind, refer to your own published work in the third
% person. That is, use ``In the previous work of Jones et al.\ [4],'' not ``In our
% previous work [4].'' If you cite your other papers that are not widely available
% (e.g., a journal paper under review), use anonymous author names in the
% citation, e.g., an author of the form ``A.\ Anonymous.''

% \subsection{Footnotes}

% Footnotes should be used sparingly.  If you do require a footnote, indicate
% footnotes with a number\footnote{Sample of the first footnote.} in the
% text. Place the footnotes at the bottom of the page on which they appear.
% Precede the footnote with a horizontal rule of 2~inches (12~picas).

% Note that footnotes are properly typeset \emph{after} punctuation
% marks.\footnote{As in this example.}

% \subsection{Figures}

% \begin{figure}
%   \centering
%   \fbox{\rule[-.5cm]{0cm}{4cm} \rule[-.5cm]{4cm}{0cm}}
%   \caption{Sample figure caption.}
% \end{figure}

% All artwork must be neat, clean, and legible. Lines should be dark enough for
% purposes of reproduction. The figure number and caption always appear after the
% figure. Place one line space before the figure caption and one line space after
% the figure. The figure caption should be lower case (except for first word and
% proper nouns); figures are numbered consecutively.

% You may use color figures.  However, it is best for the figure captions and the
% paper body to be legible if the paper is printed in either black/white or in
% color.

% \subsection{Tables}

% All tables must be centered, neat, clean and legible.  The table number and
% title always appear before the table.  See Table~\ref{sample-table}.

% Place one line space before the table title, one line space after the
% table title, and one line space after the table. The table title must
% be lower case (except for first word and proper nouns); tables are
% numbered consecutively.

% Note that publication-quality tables \emph{do not contain vertical rules.} We
% strongly suggest the use of the \verb+booktabs+ package, which allows for
% typesetting high-quality, professional tables:
% \begin{center}
%   \url{https://www.ctan.org/pkg/booktabs}
% \end{center}
% This package was used to typeset Table~\ref{sample-table}.

% \begin{table}
%   \caption{Sample table title}
%   \label{sample-table}
%   \centering
%   \begin{tabular}{lll}
%     \toprule
%     \multicolumn{2}{c}{Part}                   \\
%     \cmidrule(r){1-2}
%     Name     & Description     & Size ($\mu$m) \\
%     \midrule
%     Dendrite & Input terminal  & $\sim$100     \\
%     Axon     & Output terminal & $\sim$10      \\
%     Soma     & Cell body       & up to $10^6$  \\
%     \bottomrule
%   \end{tabular}
% \end{table}

% \section{Final instructions}

% Do not change any aspects of the formatting parameters in the style files.  In
% particular, do not modify the width or length of the rectangle the text should
% fit into, and do not change font sizes (except perhaps in the
% \textbf{References} section; see below). Please note that pages should be
% numbered.

% \section{Preparing PDF files}

% Please prepare submission files with paper size ``US Letter,'' and not, for
% example, ``A4.''

% Fonts were the main cause of problems in the past years. Your PDF file must only
% contain Type 1 or Embedded TrueType fonts. Here are a few instructions to
% achieve this.

% \begin{itemize}

% \item You should directly generate PDF files using \verb+pdflatex+.

% \item You can check which fonts a PDF files uses.  In Acrobat Reader, select the
%   menu Files$>$Document Properties$>$Fonts and select Show All Fonts. You can
%   also use the program \verb+pdffonts+ which comes with \verb+xpdf+ and is
%   available out-of-the-box on most Linux machines.

% \item The IEEE has recommendations for generating PDF files whose fonts are also
%   acceptable for NeurIPS. Please see
%   \url{http://www.emfield.org/icuwb2010/downloads/IEEE-PDF-SpecV32.pdf}

% \item \verb+xfig+ "patterned" shapes are implemented with bitmap fonts.  Use
%   "solid" shapes instead.

% \item The \verb+\bbold+ package almost always uses bitmap fonts.  You should use
%   the equivalent AMS Fonts:
% \begin{verbatim}
%    \usepackage{amsfonts}
% \end{verbatim}
% followed by, e.g., \verb+\mathbb{R}+, \verb+\mathbb{N}+, or \verb+\mathbb{C}+
% for $\mathbb{R}$, $\mathbb{N}$ or $\mathbb{C}$.  You can also use the following
% workaround for reals, natural and complex:
% \begin{verbatim}
%    \newcommand{\RR}{I\!\!R} %real numbers
%    \newcommand{\Nat}{I\!\!N} %natural numbers
%    \newcommand{\CC}{I\!\!\!\!C} %complex numbers
% \end{verbatim}
% Note that \verb+amsfonts+ is automatically loaded by the \verb+amssymb+ package.

% \end{itemize}

% If your file contains type 3 fonts or non embedded TrueType fonts, we will ask
% you to fix it.

% \subsection{Margins in \LaTeX{}}

% Most of the margin problems come from figures positioned by hand using
% \verb+\special+ or other commands. We suggest using the command
% \verb+\includegraphics+ from the \verb+graphicx+ package. Always specify the
% figure width as a multiple of the line width as in the example below:
% \begin{verbatim}
%    \usepackage[pdftex]{graphicx} ...
%    \includegraphics[width=0.8\linewidth]{myfile.pdf}
% \end{verbatim}
% See Section 4.4 in the graphics bundle documentation
% (\url{http://mirrors.ctan.org/macros/latex/required/graphics/grfguide.pdf})

% A number of width problems arise when \LaTeX{} cannot properly hyphenate a
% line. Please give LaTeX hyphenation hints using the \verb+\-+ command when
% necessary.

% \begin{ack}
% Use unnumbered first level headings for the acknowledgments. All acknowledgments
% go at the end of the paper before the list of references. Moreover, you are required to declare
% funding (financial activities supporting the submitted work) and competing interests (related financial activities outside the submitted work).
% More information about this disclosure can be found at: \url{https://neurips.cc/Conferences/2021/PaperInformation/FundingDisclosure}.

% Do {\bf not} include this section in the anonymized submission, only in the final paper. You can use the \texttt{ack} environment provided in the style file to autmoatically hide this section in the anonymized submission.
% \end{ack}

% \section*{References}

% References follow the acknowledgments. Use unnumbered first-level heading for
% the references. Any choice of citation style is acceptable as long as you are
% consistent. It is permissible to reduce the font size to \verb+small+ (9 point)
% when listing the references.
% Note that the Reference section does not count towards the page limit.
% \medskip

% {
% \small

% [1] Alexander, J.A.\ \& Mozer, M.C.\ (1995) Template-based algorithms for
% connectionist rule extraction. In G.\ Tesauro, D.S.\ Touretzky and T.K.\ Leen
% (eds.), {\it Advances in Neural Information Processing Systems 7},
% pp.\ 609--616. Cambridge, MA: MIT Press.

% [2] Bower, J.M.\ \& Beeman, D.\ (1995) {\it The Book of GENESIS: Exploring
%   Realistic Neural Models with the GEneral NEural SImulation System.}  New York:
% TELOS/Springer--Verlag.

% [3] Hasselmo, M.E., Schnell, E.\ \& Barkai, E.\ (1995) Dynamics of learning and
% recall at excitatory recurrent synapses and cholinergic modulation in rat
% hippocampal region CA3. {\it Journal of Neuroscience} {\bf 15}(7):5249-5262.
% }

% %%%%%%%%%%%%%%%%%%%%%%%%%%%%%%%%%%%%%%%%%%%%%%%%%%%%%%%%%%%%
% \section*{Checklist}

% %%% BEGIN INSTRUCTIONS %%%
% The checklist follows the references.  Please
% read the checklist guidelines carefully for information on how to answer these
% questions.  For each question, change the default \answerTODO{} to \answerYes{},
% \answerNo{}, or \answerNA{}.  You are strongly encouraged to include a {\bf
% justification to your answer}, either by referencing the appropriate section of
% your paper or providing a brief inline description.  For example:
% \begin{itemize}
%   \item Did you include the license to the code and datasets? \answerYes{See Section~\ref{gen_inst}.}
%   \item Did you include the license to the code and datasets? \answerNo{The code and the data are proprietary.}
%   \item Did you include the license to the code and datasets? \answerNA{}
% \end{itemize}
% Please do not modify the questions and only use the provided macros for your
% answers.  Note that the Checklist section does not count towards the page
% limit.  In your paper, please delete this instructions block and only keep the
% Checklist section heading above along with the questions/answers below.
% %%% END INSTRUCTIONS %%%

% \begin{enumerate}

% \item For all authors...
% \begin{enumerate}
%   \item Do the main claims made in the abstract and introduction accurately reflect the paper's contributions and scope?
%     \answerTODO{}
%   \item Did you describe the limitations of your work?
%     \answerTODO{}
%   \item Did you discuss any potential negative societal impacts of your work?
%     \answerTODO{}
%   \item Have you read the ethics review guidelines and ensured that your paper conforms to them?
%     \answerTODO{}
% \end{enumerate}

% \item If you are including theoretical results...
% \begin{enumerate}
%   \item Did you state the full set of assumptions of all theoretical results?
%     \answerTODO{}
% 	\item Did you include complete proofs of all theoretical results?
%     \answerTODO{}
% \end{enumerate}

% \item If you ran experiments...
% \begin{enumerate}
%   \item Did you include the code, data, and instructions needed to reproduce the main experimental results (either in the supplemental material or as a URL)?
%     \answerTODO{}
%   \item Did you specify all the training details (e.g., data splits, hyperparameters, how they were chosen)?
%     \answerTODO{}
% 	\item Did you report error bars (e.g., with respect to the random seed after running experiments multiple times)?
%     \answerTODO{}
% 	\item Did you include the total amount of compute and the type of resources used (e.g., type of GPUs, internal cluster, or cloud provider)?
%     \answerTODO{}
% \end{enumerate}

% \item If you are using existing assets (e.g., code, data, models) or curating/releasing new assets...
% \begin{enumerate}
%   \item If your work uses existing assets, did you cite the creators?
%     \answerTODO{}
%   \item Did you mention the license of the assets?
%     \answerTODO{}
%   \item Did you include any new assets either in the supplemental material or as a URL?
%     \answerTODO{}
%   \item Did you discuss whether and how consent was obtained from people whose data you're using/curating?
%     \answerTODO{}
%   \item Did you discuss whether the data you are using/curating contains personally identifiable information or offensive content?
%     \answerTODO{}
% \end{enumerate}

% \item If you used crowdsourcing or conducted research with human subjects...
% \begin{enumerate}
%   \item Did you include the full text of instructions given to participants and screenshots, if applicable?
%     \answerTODO{}
%   \item Did you describe any potential participant risks, with links to Institutional Review Board (IRB) approvals, if applicable?
%     \answerTODO{}
%   \item Did you include the estimated hourly wage paid to participants and the total amount spent on participant compensation?
%     \answerTODO{}
% \end{enumerate}

% \end{enumerate}

\end{document}
